\documentclass[10pt,a4paper,twoside]{article}

%%%%%%%%%%%%%%%%%%%%导言部分%%%%%%%%%%%%%%%%%%%%
%%%%%%%%%%加载宏包%%%%%%%%%%
\usepackage[heading=true]{ctex}%引入中文宏包 并使用默认样式(若不使用,无法设置标题样式)
\usepackage{geometry}%引入宏包:设置页边距
\usepackage{booktabs}%引入宏包:制作三线表
\usepackage{graphicx}%引入宏包:插入图片
\usepackage{amsmath}%引入宏包:拓展latex符号格式
\usepackage{indentfirst}%引入宏包:设置首行缩进
\usepackage{lmodern}%引入宏包:消除字体错误
\usepackage{fontspec}%引入宏包:设置英语字体
\usepackage{fancyhdr}%引入宏包:设置页眉页脚
\usepackage{gbt7714}%引入宏包:可将参考文献格式设置为GBT7714
\usepackage{caption}%引入宏包:设置图标标注
\usepackage[runin]{abstract}%引入宏包:设置摘要格式同时将摘要标题设置到摘要正文前
\usepackage[colorlinks,urlcolor=black,linkcolor=black,citecolor=black,hyperfootnotes=false]{hyperref}%引入宏包:设置引用格式,并设置引用方式
\usepackage{tikz}
\usetikzlibrary{quantikz2}
\usetikzlibrary{quotes,angles}
\usetikzlibrary{calc}
\usetikzlibrary{decorations.pathreplacing}
\usetikzlibrary{positioning}
\usepackage{physics}
\usepackage{adjustbox}
\usepackage[justification=centering,labelfont=rm,caption=false]{subfig}
\usepackage{float}
\usepackage{enumitem}
\usepackage{bm}
\usepackage{appendix}
\usepackage{amsfonts}
\usepackage{braket}
\usepackage{mathrsfs}

%%%%%%%%%%设置%%%%%%%%%%
%%%%%其他%%%%%
\geometry{a4paper,left=2cm,right=2cm,top=2.5cm,bottom=2.2cm,headheight=14pt}%【使用A4纸,左页面边距,右页面边距,上页面边距,下页面边距,页眉高度】设置页面边距
\linespread{1.25}%设置行间距为1.25倍
\setlength{\parindent}{2em}%设置首行缩进两个字节
\setmainfont{Times New Roman}%设置英文字体为新罗马字体

%%%%%页眉页脚%%%%%
\pagestyle{fancy}%启用fancy样式
\fancyhf{}%清空默认页眉页脚
\fancyfoot[C]{\thepage}%设置页脚中间为页码
\fancyhead[L]{\leftmark}%设置页眉左侧为章节标题
\fancyhead[R]{编程题二}%设置页眉右侧内容为标题
% \fancyhead[C]{XXX学报}%设置页眉中间内容为“XXX学报”
\renewcommand{\headrulewidth}{0.2pt}%设置页眉线粗细
\renewcommand{\footrulewidth}{0pt}%设置页脚线粗细

%%%%%各级标题%%%%%
\ctexset{section={format={\raggedright \zihao{-4} \mdseries},titleformat={\heiti },beforeskip=10pt,afterskip=5pt,numberformat={\setmainfont{Times New Roman}},aftername=\hspace{6pt}}}%设置一级标题格式:整体格式:左对齐 小四 不加粗声明,标题格式:黑体,前间距10磅,后间距5磅,编号格式:新罗马,编号与标题间距:6磅
\ctexset{subsection={titleformat={\heiti \zihao{5}  \mdseries},beforeskip=0pt,afterskip=0pt,numberformat={\setmainfont{Times New Roman} \zihao{5}},aftername=\hspace{5pt}}}%设置二级标题格式:标题格式:黑体 五号 不加粗声明,前间距0磅,后间距0磅,编号格式:新罗马 五号,编号与标题间距:5磅
\ctexset{subsubsection={titleformat={\kaishu \zihao{5} \mdseries},beforeskip=0pt,afterskip=0pt,numberformat={\setmainfont{Times New Roman} \zihao{5}},aftername=\hspace{5pt}}}%设置三级标题格式:标题格式:楷体 五号 不加粗声明,前间距0磅,后间距0磅,编号格式:新罗马 五号,编号与标题间距:5磅


%%%%%摘要%%%%%
\setlength{\absleftindent}{0pt}%设置摘要两端缩进
\setlength{\absrightindent}{0pt}%设置摘要两端缩进
\abslabeldelim{:}%摘要二字后加“:”
\setlength{\abstitleskip}{-2em}%设置摘要标题与摘要内容的间距


%%%%%图表标注%%%%%
\DeclareCaptionFont{songxiaoWu}{\songti \zihao{-5}}%自定义一种字体 宋体 小五 应用于图注
\DeclareCaptionFont{heixiaoWu}{\heiti \zihao{-5}}%自定义一种字体 黑体 小五 应用于表注
\captionsetup[figure]{font=songxiaoWu,labelfont=songxiaoWu,labelsep=space,skip=3pt}%设置图注字体、表现形式、上下间距
\captionsetup[table]{font=heixiaoWu,labelfont=heixiaoWu,labelsep=space,skip=3pt}%设置表注字体、表现形式、上下间距
\numberwithin{figure}{section}%设置表格按章节编号
\numberwithin{table}{section}%设置图片按章节编号

%%%%%参考文献%%%%%
\bibliographystyle{gbt7714-numerical}%设置参考文献格式
\ctexset{bibname={\heiti \zihao{-4} 参考文献}}%设置“参考文献”为黑体小四

%%%%%目录%%%%%
%%%方案tocloft%%%
%\usepackage{tocloft}
%\renewcommand{\cftbeforetoctitleskip}{0pt}%设置目录上方边距
%\renewcommand{\cftaftertoctitleskip}{0pt}%设置目录下方边距
%\renewcommand{\cfttoctitlefont}{\hfill \kaishu \zihao{4}}%设置目录二字格式
%\renewcommand{\cftaftertoctitle}{\hfill}%设置目录居中
%\renewcommand{\listfigurename}{图片目录}
%\renewcommand{\listtablename}{表格目录}
%\setcounter{tocdepth}{2}%设置目录级数

%%%方案titlesec,titletoc%%%
%\usepackage{titlesec,titletoc}%引入宏包:设置目录
%\renewcommand{\contentsname}{\centering 目\quad 录}%设置目录二字的格式
%\dottedcontents{section}[2em]{}{2em}{2em}%标题级数%标题至左侧距离%预定义标题格式%序号与标题间距%点点的间距
%\titlecontents{section}[2em]{}{\thecontentslabel[]{2em}}{}{\titlerule*[2ex]{-}\thecontentspage}%标题级数(可为各级标题 也可为图表)%标题距左侧间距%预定义标题格式字体字号%有序号的目录类型%无序号的目录类型%填充与页码格式

%%%%%导言格式%%%%%
\makeatletter
% 重定义\@maketitle,去掉标题后的垂直间距
\renewcommand{\@maketitle}{
  \newpage
  \null
  \begin{center}
    {\@title \par}%标题
    \vskip 1em %标题与作者之间的间距
    {\@author \par}%作者
    \vskip 0em %作者与日期之间的间距
    {\@date \par}%日期
  \end{center}
  \par %移除默认的\vspace{\baselineskip}
}
\makeatother

%%%%%导言内容%%%%%
\title{\heiti \zihao{-2} 空气质量四分类任务:经典与量子机器学习模型设计与性能分析}%标题内容
\author{\kaishu \zihao{4} 作者\textsuperscript{1},作者\textsuperscript{2}}%作者姓名
\date{{\kaishu \zihao{5} (1.作者详细单位,省 市 邮编;2.作者详细单位,省 市 邮编;)}\\
\textsuperscript{1}~{\kaishu \zihao{5} 作者详细单位,省 市 邮编;} 
\textsuperscript{2}~{\kaishu \zihao{5} 作者详细单位,省 市 邮编;}}%将作者单位信息塞入日期栏


%%%%%%%%%%%%%%%%%%%%文章部分%%%%%%%%%%%%%%%%%%%%
\begin{document}%开启文章

%%%%%%%%%%正文%%%%%%%%%%
%%%%%字体字号%%%%%
\songti%设置正文字体为宋体
\zihao{5}%设置正文字号

\maketitle%使导言内容显示

%%%%%或将作者单位信息在后部单独设置并手动调整间距%%%%%
%\vspace{-5pt}%减小导言与后文的间距
%设置作者单位
%\begin{center}
    %{\kaishu \zihao{5} (1.作者详细单位,省 市 邮编;2.作者详细单位,省 市 邮编;)}\\
    %\textsuperscript{1}~{\kaishu \zihao{5} 作者详细单位,省 市 邮编;} 
    %\textsuperscript{2}~{\kaishu \zihao{5} 作者详细单位,省 市 邮编;}
%\end{center}

%%%%%文章及作者信息脚注%%%%%
% \renewcommand{\thefootnote}{}%设置脚注标号为空
% \footnotetext{\zihao{-5} {\heiti 收稿日期:}xxxx-xx-xx}%设置收稿日期脚注
% \footnotetext{\zihao{-5} {\heiti 作者简介:}{\songti 姓  名(出生年-),性别,籍贯(注明市县),职称,学位,主要研究方向. E-mail:}}%设置作者信息脚注
% \footnotetext{\zihao{-5} {\heiti 作者简介:}{\songti 姓  名(出生年-),性别,籍贯(注明市县),职称,学位,主要研究方向. E-mail:}}%设置作者信息脚注

%%%%%摘要%%%%%
\renewcommand{\abstractname}{\zihao{-5} \heiti \mdseries 摘\quad 要}
\begin{abstract}
    \zihao{-5}
    摘要内容。概括地陈述论文研究的目的、方法、结果、结论,要求200~300字。应排除本学科领域已成为常识的内容;不要把应在引言中出现的内容写入摘要,不引用参考文献;不要对论文内容作诠释和评论。不得简单重复题名中已有的信息。用第三人称,不使用“本文”、“作者”等作为主语。使用规范化的名词术语,新术语或尚无合适的汉文术语的,可用原文或译出后加括号注明。除了无法变通之外,一般不用数学公式和化学结构式,不出现插图、表格。缩略语、略称、代号,除了相邻专业的读者也能清楚理解的以外,在首次出现时必须加括号说明。结构严谨,表达简明,语义确切。
    \par \noindent{{\heiti 关键词:}{关键词1;关键词2;关键词3;关键词4}}
    \par \noindent{{\heiti 中图分类号:}{作者本人填写} \qquad {\heiti 文献标识码:}{A}}
\end{abstract}

%%%%%英文部分%%%%%
% \begin{center}
%     {\bf \zihao{3} How to use {\LaTeX} to edit an essay elegantly\par\vspace*{10pt}}
%     {\zihao{-4} NAME Name\textsuperscript{1},\ NAME Name\textsuperscript{2}}\\
%     {\zihao{5} (1.Department, City, City Zip Code, China;\ 2.Department, City, City Zip Code, China;)}\\
%     \textsuperscript{1}~{\zihao{5} Department, City, City Zip Code, China;} \
%     \textsuperscript{2}~{\zihao{5} Department, City, City Zip Code, China;} 
% \end{center} 
% \renewcommand{\abstractname}{\zihao{5} \bf Abstract}
% \begin{abstract}
%     \zihao{5}
%     Purpose purpose purpose purpose purpose purpose purpose purpose purpose purpose purpose purpose purpose purpose purpose purpose purpose purpose purpose purpose purpose purpose purpose purpose purpose purpose purpose purpose purpose purpose purpose purpose. Method method method method method method method method method method method method method method method method method method. Result result result result result result result result result result result result result result result result result result result result result result result result result result result. Conclusion conclusion conclusion conclusion conclusion conclusion conclusion conclusion conclusion conclusion conclusion conclusion conclusion conclusion conclusion conclusion conclusion conclusion conclusion conclusion.
%     \par \noindent{\textbf{Keywords:}{keyword1; keyword2; keyword3; keyword4}}
% \end{abstract}
% \qquad 两个空格   \quad 一个空格    \ 三分之一个空格    \;七分之二个空格    \, 六分之一个空格

%%%%%引言部分%%%%%
\par{引言内容。引言作为论文的开场白,应以简短的篇幅介绍论文的写作背景和目的,以及相关领域内前人所做的工作和研究概况,说明本研究与前人工作的关系,目前研究的热点、存在的问题及作者工作的意义。1、开门见山,不绕圈子。避免大篇幅地讲述历史渊源和立题研究过程。2、言简意赅,突出重点。不应过多叙述同行熟知的及教科书中的常识性内容,确有必要提及他人的研究成果和基本原理时,只需以引用参考文献的形势标出即可。在引言中提示本文的工作和观点时,意思应明确,语言应简练。3、引言的内容不要与摘要雷同,也不是摘要的注释。4、引言最好不要有插图、列表和数学公式。}

%%%%%目录%%%%%
%\newpage
%\tableofcontents%章节目录
%\listoffigures%图片目录
%\listoftables%表格目录
%\newpage

%%%%%公式设置%%%%%
\numberwithin{equation}{section}%配置公式按章节编号
\setlength{\abovedisplayskip}{2pt}%设置公式前间距
\setlength{\belowdisplayskip}{2pt}%设置公式后间距

%%%%%%%%%%%%%%%正文%%%%%%%%%%%%%%%

\section{概述}

空气污染是全球性环境问题,对人类健康和生态系统产生了深远的影响。准确评估空气质量并预测污染等级对于制定环境保护政策和改善居民生活质量至关重要。本次比赛旨在利用量子机器学习技术,基于多维度的环境数据,预测空气质量等级,为城市环境管理提供决策支持。

我们的代码结构如下:

\begin{itemize}
	\item \texttt{Data}文件夹:存放数据集\texttt{train\_data.csv}和\texttt{test\_data.csv}。
	\item \texttt{Utils.py}:实现数据读取、预处理等的工具类和函数。
	\item \texttt{ClassicalModel.py}:实现经典机器学习模型的网络框架。
	\item \texttt{QuantumModel.py}:实现量子机器学习模型的网络框架。
	\item \texttt{Trainer.py}:实现模型训练和评估。
	\item \texttt{Main.py}:主程序,负责调用各个模块,进行数据加载、模型训练和评估。
\end{itemize}

%%%%%%%%%%%%%%%%%%%%%%%%%%%%%%%%%%%%%%%%%%%%%%%%%%%%%%%%%%%%%%%%%%%%%%%%%%%%%%%%%%%%%%%%%%%%%%%%%%%%%%%%%%%%%%%%%%%%%%%%%%%%%%%

\section{数据探索与预处理}

本任务中提供了\texttt{train\_data.csv}与\texttt{test\_data.csv}两个数据集。两个文件具有相同的结构,从左到右依次为:温度(${}^\circ C$)、湿度(\%)、$PM_{2.5}$浓度($\mu g/m^3$)、$PM_{10}$浓度($\mu g/m^3$)、$NO_2$浓度($ppb$)、$SO_2$浓度($ppb$)、$CO$浓度($ppm$)、到最近工业区的距离($km$)、人口密度(人$/$$km^2$)、空气质量。其中,前九列是特征,最后一列即空气质量是标签,共有四类:Good、Moderate、Poor和Hazardous,在之后我们会将其映射为$0\sim3$。

为了读取文件,我们在\texttt{Utils.py}文件中实现了\texttt{CSVReader}类。该类的构造函数读入一个数据集文件路径,自动加载文件中所有数据,包括列名、数据维度、缺失值等。通过调用函数,也可以查看读入数据的统计信息,例如均值和方差等。这个类中有以下方法:

\begin{itemize}
	\item \texttt{\_\_init\_\_(self, filePath: str)}:构造函数,传入数据集文件路径,自动读取数据集。
	\item \texttt{\_\_loadData\_\_(self)}:读取数据集文件,没有返回值。会被构造函数自动调用,用户不得手动调用。
	\item \texttt{getData(self)}:以二元组返回数据集的特征和标签。
	\item \texttt{checkBasicInfo(self)}:输出检查数据集的基本信息,包括维度、缺失值等。
	\item \texttt{showStatistics(self)}:输出数据集的统计信息,包括均值、方差等。
	\item \texttt{read(filePath: str)}:静态方法,传入数据集文件路径,返回数据集特征和标签。用于快速读取数据集。
\end{itemize}

一般而言,为了模型能够更好地学习特征,我们会对数据进行预处理。由于数据量较小,简单的预处理即可完成需求。我们在\texttt{Utils.py}文件中实现了\texttt{Math}静态类。这个类中只有一个静态函数:

\begin{itemize}
	\item \texttt{normalize(data: np.ndarray)}:对数据进行归一化处理,返回归一化后的数据。
\end{itemize}

这是为了保证不会修改CSVReader类中的数据。在\texttt{Trainer.py}文件中,\texttt{Trainer}类读入数据后会自动调用\texttt{Math.normalize}函数对数据进行归一化处理。

%%%%%%%%%%%%%%%%%%%%%%%%%%%%%%%%%%%%%%%%%%%%%%%%%%%%%%%%%%%%%%%%%%%%%%%%%%%%%%%%%%%%%%%%%%%%%%%%%%%%%%%%%%%%%%%%%%%%%%%%%%%%%%%

\section{经典神经网络模型}

\subsection{算法原理}
我们实现了两种经典神经网络结构作为基准模型:
\begin{itemize}
\item \textbf{输入维度适配}:将9维空气质量特征(温度、湿度、PM2.5等)归一化至统一尺度,消除量纲影响

\item \textbf{标签编码}:将4类空气质量标签(Good/Moderate/Poor/Hazardous)映射为数值型0-3

\item \textbf{数据流设计}:通过CSVReader类实现自动化数据加载与统计验证
\end{itemize}
\subsubsection{残差网络(SmallResNet)}
\begin{itemize}
    \item \textbf{残差连接}:通过跳跃连接解决梯度消失问题
    \begin{equation}
        y = \mathcal{F}(x) + x
    \end{equation}
    
    \item \textbf{网络结构}:
    \begin{itemize}
        \item 输入层:9维$\rightarrow$8维全连接
        \item 残差块:8维$\xrightarrow{ReLU}$4维$\rightarrow$8维
        \item 输出层:8维$\rightarrow$4维+Softmax
    \end{itemize}
    
\end{itemize}

\subsubsection{多层感知器(SmallMLP)} 
\begin{itemize}
    \item \textbf{前馈结构}:
    \begin{equation}
        \mathbf{h}_l = \text{ReLU}(\mathbf{W}_l\mathbf{h}_{l-1} + \mathbf{b}_l)
    \end{equation}
    
    \item \textbf{网络结构}:
    \begin{itemize}
        \item 输入层:9维$\rightarrow$8维
        \item 隐藏层:8维$\rightarrow$6维
        \item 输出层:6维$\rightarrow$4维+Softmax
    \end{itemize}

\end{itemize}

\subsection{实现细节}

我们实现的一个MLP网络结构如图\ref{fig:MLP}所示。该网络包含一个输入层、两个隐藏层和一个输出层。输入层有9个节点,分别对应9个特征;第一个隐藏层有8个节点,第二个隐藏层有6个节点;输出层有4个节点,分别对应四类标签。每一层的神经元之间是全连接的。

\begin{figure}[tb]
    \centering 
    \begin{tikzpicture}[
        neuron/.style={circle, draw=black, minimum size=5mm, inner sep=0},
        >=stealth,
        node distance=1.5cm
    ]
        % 定义层参数
        \def\layers{
            Input/9/0, 
            Hidden1/8/3, 
            Hidden2/6/6, 
            Output/4/9
        }
        
        % 计算最大高度(用于居中对齐)
        \pgfmathsetmacro{\maxheight}{max(9,8,6,4)*0.8}
        
        % 绘制神经元和标签
        \foreach \layername/\layercount/\x in \layers {
            % 绘制神经元(垂直居中)
            \foreach \i in {1,...,\layercount} {
                \node[neuron] (\layername-\i) at (\x, {\i*0.8 - \layercount*0.4 - 0.4}) {};
            }
            
            % 记录最下方神经元位置(第一列)
            \ifnum\x=0
                \coordinate (labelanchor) at (0, {-\layercount*0.4 - 0.6});
            \fi
            
            % 绘制标签(与第一列底部对齐)
            \node[anchor=north, align=center] at (\x, {-\layercount*0.4 - 0.6}) {\layername\\(\layercount\ 节点)};
        }
    
        % 绘制全连接线
        \foreach \i in {1,...,9}{
            \foreach \j in {1,...,8}{
                \draw[->, gray!80] (Input-\i) -- (Hidden1-\j);
            }
        }
        \foreach \i in {1,...,8}{
            \foreach \j in {1,...,6}{
                \draw[->, gray!80] (Hidden1-\i) -- (Hidden2-\j);
            }
        }
        \foreach \i in {1,...,6}{
            \foreach \j in {1,...,4}{
                \draw[->, gray!80] (Hidden2-\i) -- (Output-\j);
            }
        }
    \end{tikzpicture}	
    \caption{MLP网络结构}
    \label{fig:MLP}
\end{figure}

\subsection{性能评估}

%%%%%%%%%%%%%%%%%%%%%%%%%%%%%%%%%%%%%%%%%%%%%%%%%%%%%%%%%%%%%%%%%%%%%%%%%%%%%%%%%%%%%%%%%%%%%%%%%%%%%%%%%%%%%%%%%%%%%%%%%%%%%%%

\section{量子机器学习模型}

\subsection{算法原理}
QMLP架构包含三大创新模块:

\subsubsection{容错输入编码}
\begin{itemize}
    \item 单比特旋转门实现线性编码:
    \begin{equation}
        S(\mathbf{x}_i) = \bigotimes_{k=0}^{n-1} RX(x_i^k)
    \end{equation}
    
    \item 错误隔离机制:量子门错误仅影响单个量子比特
\end{itemize}

\subsubsection{可调非线性单元}
\begin{itemize}
    \item 动态重上传结构:
    \begin{equation}
        R_i(\mathbf{x}) = \text{RX}(\text{ReLU}(\mathbf{x}))
    \end{equation}
    
    \item 支持多种非线性组合模式:
    \begin{itemize}
        \item RX/RY/RZ旋转门混合
        \item 经典非线性函数嵌入
    \end{itemize}
\end{itemize}

\subsubsection{参数化纠缠层} 
\begin{itemize}
    \item 可训练双比特门:
    \begin{equation}
        U(\theta) = \prod_{k=1}^L \text{CRX}(\theta_k)
    \end{equation}
    
    \item 动态纠缠调控:相比固定CNOT门减少$3\times$参数
\end{itemize}

\begin{table}[H]
    \centering
    \caption{QMLP量子优势}
    \begin{tabular}{lc}
        \toprule
        指标 & 提升幅度 \\
        \midrule
        参数量 & 减少3$\times$ \\
        量子门数量 & 减少2$\times$ \\
        噪声鲁棒性 & 错误率降低30\% \\
        分类准确率 & 提高10\% \\
        \bottomrule
    \end{tabular}
\end{table}
\subsection{实现细节}

\subsection{性能评估}

%%%%%%%%%%%%%%%%%%%%%%%%%%%%%%%%%%%%%%%%%%%%%%%%%%%%%%%%%%%%%%%%%%%%%%%%%%%%%%%%%%%%%%%%%%%%%%%%%%%%%%%%%%%%%%%%%%%%%%%%%%%%%%%

\section{性能对比与分析}

\subsection{性能对比}

\subsection{差异分析}



%插入表格(三线表)
% \begin{table}[htbp]
%     \centering
%     \caption{表题}
%     \begin{tabular}{ccccc}%{{5}{c}}

%         \toprule
%         {组别} & {物理量1/单位} & {物理量2/单位} & {物理量3/单位} & {物理量4/单位} \\
%         \midrule
%         A&{}&{}&{}&{}\\
%         B&{}&{}&{}&{}\\
%         C&{}&{}&{}&{}\\
%         D&{}&{}&{}&{}\\
%         \bottomrule
%     \end{tabular}
% \end{table}


%插入图片
% \begin{figure}[htbp]
%     \centering 
%     \includegraphics[height=5.99cm,width=8.25cm]{E:/engineer file/latex/learn1/photo/test.png}
%     \caption{图注}
% \end{figure}
%table和figure为浮动体






%%%%%列出参考文献%%%%%
\zihao{-5}%设置参考文献字号
\bibliography{refs}%调用bib文件 加入参考文献
%知网复制出的latex代码比工大图书馆复制出的latex代码好用

%%%%%%%%%%%%%%%附录部分%%%%%%%%%%%%%%%
\newpage
\appendix%设置附录
\zihao{-5}%设置附录字号
\section*{附录一:}




\end{document}